\documentclass[a4paper,11pt]{article}
\usepackage{amsmath}
\usepackage{amssymb}
\usepackage[margin = 1.0in]{geometry}
% define the title
\author{Sam Cook}
\title{Intro to Modern Algebra \\Homework 3}
\date{October 19th, 2017}
\begin{document}
% generates the title
\maketitle
\section{Section 3.3 \#8}
\paragraph{Problem}
Let $\mathbb{Q}(\sqrt{2})$ be as in Exercise 39 of Section 3.1 Prove that the function $f:\mathbb{Q}(\sqrt{2})  \rightarrow \mathbb{Q}(\sqrt{2})$
given by $f(a + b\sqrt{2}) = a - b\sqrt{2}$ is an isomorphism.
\paragraph{Solution}

In order for $f$ to be an isomorphism, it must be a bijective homomorphism.

First, to be bijective, it must be both injective and surjective.
Take $f(a), f(b) \epsilon \mathbb{Q}(\sqrt{2})$ such that $f(a) = f(b)$. Then, for some $k,l,m,n \epsilon \mathbb{Q}, f(k+l\sqrt{2}) = f(m + n\sqrt{2})$. This implies that $k-l\sqrt{2} = m - n\sqrt{2}$, by the definition of $f$. We can group the rational and irrational parts together to get $k-l\sqrt{2} = m - n\sqrt{2} \Rightarrow (k-m) = (l-n)\sqrt{2}$. But $(k-m)$ is rational, and $(l-n)\sqrt{2}$ is irrational. The only way this could be possible is if $(k-m)=0$ and $(l-n)=0$, since $(0-0) = (0-0)\sqrt{2}$. This implies that $k=m$ and $l=n$. Therefore, $k+l\sqrt{2} = m + n\sqrt{2}$, or $a=b$ and $f$ is an injective function.

Now, consider $c + d\sqrt{2}  \epsilon \mathbb{Q}(\sqrt{2})$ for any $c,d \epsilon \mathbb{Q}$. $c + d\sqrt{2}$ is mapped to from $c - d\sqrt{2}$, which can also be written as $c + (-d)\sqrt{2}$, since $f(c - d\sqrt{2}) = c + d\sqrt{2}$. Therefore, $f$ is surjective, and as a result of that, bijective.\\

Now, we must prove that $f$ is a homomorphism.
Consider  $f((k+l\sqrt{2}) * (m + n\sqrt{2}))$ with $(k+l\sqrt{2}), (m + n\sqrt{2}) \epsilon \mathbb{Q}(\sqrt{2})$. We can multiply the terms to get $f(km + kn\sqrt{2} + ml\sqrt{2} + nl)$, which simplifies to $f((km + nl) + (kn + ml)\sqrt{2})$. By applying the function $f$ and distributive laws,  $f((km + nl) + (kn + ml)\sqrt{2}) = (km + nl) - (kn + ml)\sqrt{2}$. Now consider $f(k+l\sqrt{2}) * f(m+n\sqrt{2})$. By appliying $f$, we get $(k -l\sqrt{2}) * (m - n\sqrt{2})$. We  can use distributive laws to get $km - kn\sqrt{2} - ml\sqrt{2} +2nl = (km + 2nl) - (kn + ml)\sqrt{2} = f((k+l\sqrt{2}) * (m + l\sqrt{2}))$. Therefore, $f$ preserves multiplication

Now consider $f((k+l\sqrt{2}) + (m + n\sqrt{2}))$ This simplifies to $f((k+m) + (n+l)\sqrt{2})$.Applying $f$, $f((k+m) + (n+l)\sqrt{2}) = (k+m) - (n+l)\sqrt{2}$.  Next consider $f(k + l\sqrt{2}) + f(m + n\sqrt{2})$. Applying $f$, $f(k + l\sqrt{2}) + f(m + n\sqrt{2}) = (k-l\sqrt{2}) + (m-\sqrt{2}) = (k+m) - (n+l)\sqrt{2}$, by associativity. But $(k+m) - (n+l)\sqrt{2} = f((k+l\sqrt{2}) + (m+n\sqrt{2})$ and therefore $f$ preserves addition.
 Since $f$ preserves addition and multiplication, it is a homomorphism, and since it is a bijective homomorphism, it is an isomorphism.

\section{Section 3.3\# 12e}
\paragraph{Problem} Is the following function a homeomorphism or not?
\begin{equation}
f: \mathbb{Z}_{12} \rightarrow \mathbb{Z}_4 
\end{equation}
defined by $f([x]_{12}) = [x]_4$, where $[u]_4$ denotes the class of the integer $u$ in $\mathbb{Z}_n$
\paragraph{Solution}
First, we need to prove that this function is well defined. To do this, consider $[a]_{12}$ and $[a+k12]_{12}, k\epsilon\mathbb{Z}$ such that $a, a+12\ epsilon \mathbb{Z}_{12}$. It is clear that $[a]_{12} = [a+12]_{12}$, since these are equivalence classes in $\mathbb{Z}_{12}$. Now consider $f([a]_{12})$ and $f([a+12]_{12})$. Then applying $f$, $f([a]_{12}) = [a]_4$ and $f([a+12]_{12}) = [a+12]_4$. But $[a+12]_4 = [a]_4 + [12]_4 = [a]_4 + [0]_4 = [a]_4$, since it is an equivalence class. Then $f([a]_{12}) = f([a+12]_{12}) $, and $f$ is well defined.

Now, for $f$ to be a homomorphism, it must preserve addition and multiplication. Consider $f([x]_{12} + [y]_{12})$, $x,y \epsilon \mathbb{Z}_{12}$. Then $f([x]_{12} + [y]_{12}) = f([x+y]_{12}) = [x+y]_{4}$, by the definitions of equivalence classes and $f$. Now consider $f([x]_{12}) + f([y]_{12}) = [x]_4 + [y]_4 = [x+y]_4 = f([x]_{12} + [y]_{12})$, by the same properties of equivalence classes. Since $f([x]_{12} + [y]_{12}) = f([x]_{12}) + f([y]_{12})$, $f$ preserves addition.

 To check if $f$ preserves multiplication, consider $f([x]_{12}*[y]_{12})$. We can use rules of equivalence classes and apply the function $f$ to get $f([x]_{12}*[y]_{12}) = f([xy]_{12}) = [xy]_4$. Now consider $f([x]_{12}) * f([y]_{12})$. By applying $f$, we get $f([x]_{12}) * f([y]_{12}) = [x]_4 * [y]_4 = [xy]_4 = f([x]_{12}*[y]_{12})$, and therefore $f$ preserves multiplication.

Since $f$ is well-defined and preserves multiplication and addition, it is homomorphism.


\section{Section 3.3\#30}
\paragraph{Problem}
Let $f:R\rightarrow S$ be a homomorphism of rings and let $K = \{r\epsilon R | f(r) = 0_S\}$.
Prove that K is a subring of R.
\paragraph{Solution}

Since $f$ is a homomorphism of rings, $0_R \epsilon R$, $0_S \epsilon S$, and $f(0_R) = 0_S$. Therefore, $0_R \epsilon K$, and $K \neq \emptyset$.

Now take $a, b \epsilon K$. By the definition of $K$, $f(a) = 0_S$ and $f(b) = 0_S$. Subtracting one from the other, $f(a) - f(b) = 0_S - 0_S = 0_S$. Since $f$ is a ring homomorphism, it preserves subtraction, and $f(a) - f(b) = f(a - b) = 0_S$. This implies that $a-b \epsilon K$, and therefore $K$ is closed under subtraction.

To proved that $K$ is closed under multiplication, consider the same $a,b$ from above. Then $f(a) * f(b) = 0_S * 0_S = 0_S$. Again, since $f$ is a ring homomorphism, it preserves multiplication and $f(a) * f(b) = f(a*b) = 0_S$. Therefore, $ab \epsilon K$ and $K$ is closed under multiplication. 

Since $K$ is a nonempty subset closed under subtraction and multiplication, it is a subring of $R$.

\section{Section 3.3 \#38}
\paragraph{Problem}
Let $F$ be a field and $f:F \rightarrow R$ a homomorphism of rings.
\subparagraph{(a)}
If there is a nonzero element $c$ of $F$ such that $f(c) = 0_R$, prove that $f$ is the zero homomorphism (that is, $f(x) = 0_R$ for every $x \epsilon F$). [Hint: $c^{-1}$ exists (Why?). If $x \epsilon F$, consider $F(xcc^{-1})$.]

\subparagraph{(b)}
Prove that $f$ is either injective or the zero homomorphism. [Hint: If $f$ is not the zero homomorphism and $f(a)= f(b)$, then $f(a-b) = 0_R$.]

\paragraph{Solution}
\subparagraph{(a)}
Take $c \epsilon F, c \neq 0, f(c) = 0_R$ and take any $x \epsilon F$. Note that since $F$ is a field, $c^{-1}$ exists. Now consider the product$f(x)*f(c)*f(c^{-1}) = 0_R$. We know that the solution is $0_R$, since $f(c) = 0_R$, and $0_R$ multiplied by anything is $0_R$. Since $f$ is a ring homomorphism, $f(x)*f(c)*f(c^{-1}) = f(xcc^{-1}) = f(x*1) = f(x) = 0_R$. This implies that for any element $x \epsilon F, f(x) = 0_R$, and therefore $f$ is the zero homomorphism.

\subparagraph{(b)}
If there is a nonzero element $c$ of $F$ such that $f(c) = 0_R$, then $f$ is the zero homomorphism, from part (a) above. Assume that there there is no such element $c$ of $F$. Now assume that for $a,b \epsilon F$, $f(a) = f(b)$. Then $f(a) - f(b) = f(a-b) = 0_R$, since $f$ is a ring homomorphism. Since $F$ is a field and $f$ is a ring homomorphism, $f(a-b) = 0_R$ implies that $a-b = 0_F$, which can be simplified to $a-b = 0_F \Rightarrow a = b + 0_F \Rightarrow a = b$. Therefore, $f$ is injective.

Now, if there is a nonzero element $c$ of $F$ such that $f(c) = 0_R$, then $f$ is the zero homomorphism, and if there is not, $f$ is injective.

\section{Extra Problem}
\paragraph{Problem}
Define an equivalence relation on the set of all rings by defining a ring $R$ to be equivalent to a ring $S$ if there is a ring isomorphism $f: R \rightarrow S$, i.e. $R$ is isomorphic to $S$, $R \simeq S$. 
Show that this is an equivalence relation by showing\\
$\bullet R \simeq R$ for all rings $R$\\
$\bullet$ If $R \simeq S$ for rings $R$ and $S$, then $S \simeq R$\\
$\bullet$ If $R \simeq S$ and $S \simeq T$ for rings $R$, $S$, and $T$, then $R \simeq T$ (See problem \#27)

\paragraph{Solution\\}  
$\bullet$ Let $f: R \rightarrow R$ be defined for $a \epsilon R$ as  $f(a) = a$, the identity function. 

The identity function is bijective, but this is easy to prove. To prove injectivity, assume that $f(x) = f(y)$ for some $x,y \epsilon R$. By the definition of $f$, $f(x) = f(y) \Rightarrow x = y$, and therefore $f$ is injective. To prove surjectivity, take $f(z) \epsilon R$. $f(z)$ is mapped to by $z \epsilon R$, and thus $f$ is surjective. Since $f$ is surjective and injective, it is bijective.

Now to prove that $f$ is a homomorphism, we must prove that it preserves addition and multiplication. 

Consider $f(x + y)$ such that $x,y \epsilon R$. Then $f(x+y) = x + y$. Now consider $f(x)$ and $f(y)$. Well $f(x) = x$ and $f(y) = y$, so $f(x+y) = x + y = f(x) + f(y)$. Therefore $f$ preserves addition.

Next, consider $f(xy) = xy$, $f(x) = x$, and $f(y) = y$. Then $f(xy) = xy = f(x)f(y)$, and therefore $f$ preserves multiplication.

Thus, $f: R \rightarrow R$ is an isomorphism for all rings $R$.\\

$\bullet$ Assume there is some function $f: R \rightarrow S$ such that $f$ is an isomorphism. Consider $g: S \rightarrow R$ such that $g$ is the inverse of $f$, $f^{-1}$. We know that the inverse of a bijective function is bijective, so this means that $g$ is bijective. 

Now, we need to prove that $g$ is a homomorphism. By the definition of inverse functions, $g$ is $f^{-1}$, $g(f(x)) = x$, for some $x \epsilon R$.

Consider $g(c + d)$ for some $c,d \epsilon S$. We know that $c$ and $d$ are mapped to by $f$, since $f$ is onto, so let $f(a) = c$ and $f(b) = d$ for some $a,b \epsilon R$. Then $g(c + d) = g(f(a) + f(b)) = g(f(a+b))$, since $f$ is isomorphic. Applying $g$, $g(f(a+b)) = a + b = g(f(a)) + g(f(b)) = g(c) + g(d)$, by applying $g$ and substituting. Therefore $g$ preserves addition.

Next, consider $g(c * d)$ for some $c,d \epsilon S$. Similarly, we know that $c$ and $d$ are mapped to by $f$, since $f$ is onto, so let $f(a) = c$ and $f(b) = d$ for some $a,b \epsilon R$. Then $g(c * d) = g(f(a) * f(b))$. Again, since $f$ is isomorphic, it preserves multiplication and $g(f(a)*f(b)) = g(f(ab)) = ab = g(f(a))*g(f(b)) = g(c) * g(d)$, by applying $g$ and substituting. Thus, $g$ preserves multiplication. 

Since $g$ is a bijection that preserves addition and multiplication, $g: S \rightarrow R$ is an isomorphism.\\

$\bullet$ Assume that $f: R \rightarrow S$ and $g: S \rightarrow T$ are isomorphisms. Then $f$ and $g$ are bijective, and we must show that $g \circ f$ is bijective.

Take some $a \epsilon T$. Since $g$ is bijective, $a$ is mapped to by an element of $S$, so let $g(b) = a, b \epsilon S$. But since $f$ is bijective, there is an element $c \epsilon R$ such that $f(c) = b$. Then we can rewrite $a$ as $g(f(c))$. Thus, for any element in $T$, there is some element in $R$ that maps to it, and the $g \circ f$ is surjective.

Now, we must prove that $g\circ f$ is an homomorphism. Consider $g(f(x+y))$, for some $x,y\epsilon R$ and $f(x+y)\epsilon S$. Since $f$ and $g$ are isomorphisms, $g(f(x+y)) = g(f(x)+f(y)) = g(f(x)) + g(f(y))$, so $g\circ f$ respects addition. 

Now, consider $g(f(x*y))$. Again, since $f$ and $g$ are isomorphisms, they respect multiplication, $g(f(x*y)) = g(f(x)*f(y)) = g(f(x)) * g(f(y))$, and $g\circ f$ respects multiplication.

Therefore, since $g \circ f$ preserves multiplication and addition, and is a bijection from $R \rightarrow T$, $R$ and $T$ are isomorphic.

\end{document}
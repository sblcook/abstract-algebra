\documentclass{article}

\usepackage{fancyhdr}
\usepackage{extramarks}
\usepackage{amsmath}
\usepackage{amsthm}
\usepackage{amsfonts}
\usepackage{tikz}
\usepackage[plain]{algorithm}
\usepackage{algpseudocode}
\usepackage{enumitem}
\usepackage{polynom}

\usetikzlibrary{automata,positioning}

%
% Basic Document Settings
%

\topmargin=-0.45in
\evensidemargin=0in
\oddsidemargin=0in
\textwidth=6.5in
\textheight=9.0in
\headsep=0.25in

\linespread{1.1}

\pagestyle{fancy}
\lhead{\hmwkAuthorName}
\chead{\hmwkClass}
\rhead{\firstxmark}
\lfoot{\lastxmark}
\cfoot{\thepage}

\renewcommand\headrulewidth{0.4pt}
\renewcommand\footrulewidth{0.4pt}

\setlength\parindent{0pt}

%
% Create Problem Sections
%

\newcommand{\enterProblemHeader}[1]{
    \nobreak\extramarks{}{Problem \arabic{#1} continued on next page\ldots}\nobreak{}
    \nobreak\extramarks{Problem \arabic{#1} (continued)}{Problem \arabic{#1} continued on next page\ldots}\nobreak{}
}

\newcommand{\exitProblemHeader}[1]{
    \nobreak\extramarks{Problem \arabic{#1} (continued)}{Problem \arabic{#1} continued on next page\ldots}\nobreak{}
    \stepcounter{#1}
    \nobreak\extramarks{Problem \arabic{#1}}{}\nobreak{}
}

\setcounter{secnumdepth}{0}
\newcounter{partCounter}
\newcounter{homeworkProblemCounter}
\setcounter{homeworkProblemCounter}{1}
\nobreak\extramarks{Problem \arabic{homeworkProblemCounter}}{}\nobreak{}

%
% Homework Problem Environment
%
% This environment takes an optional argument. When given, it will adjust the
% problem counter. This is useful for when the problems given for your
% assignment aren't sequential. See the last 3 problems of this template for an
% example.
%
\newenvironment{homeworkProblem}[1][-1]{
    \ifnum#1>0
        \setcounter{homeworkProblemCounter}{#1}
    \fi
    \section{Problem \arabic{homeworkProblemCounter}}
    \setcounter{partCounter}{1}
    \enterProblemHeader{homeworkProblemCounter}
}{
    \exitProblemHeader{homeworkProblemCounter}
}

%
% Homework Details
%   - Title
%   - Due date
%   - Class
%   - Section/Time
%   - Instructor
%   - Author
%

\newcommand{\hmwkTitle}{Homework\ \#2}
\newcommand{\hmwkDueDate}{November 2}
\newcommand{\hmwkClass}{Intro to Modern Algebra}
\newcommand{\hmwkClassTime}{9:30-11:00}
\newcommand{\hmwkClassInstructor}{Professor Lorenz}
\newcommand{\hmwkAuthorName}{\textbf{Sam Cook}}

%
% Title Page
%

\title{
    \vspace{2in}
    \textmd{\textbf{\hmwkClass:\ \hmwkTitle}}\\
    \normalsize\vspace{0.1in}\small{Due\ on\ \hmwkDueDate\ at 9:30am}\\
    \vspace{0.1in}\large{\textit{\hmwkClassInstructor\ \hmwkClassTime}}
    \vspace{3in}
}

\author{\hmwkAuthorName}
\date{}

\renewcommand{\part}[1]{\textbf{\large Part \Alph{partCounter}}\stepcounter{partCounter}\\}

%
% Various Helper Commands
%


% Useful for algorithms
\newcommand{\alg}[1]{\textsc{\bfseries \footnotesize #1}}

% For derivatives
\newcommand{\deriv}[1]{\frac{\mathrm{d}}{\mathrm{d}x} (#1)}

% For partial derivatives
\newcommand{\pderiv}[2]{\frac{\partial}{\partial #1} (#2)}

% Integral dx
\newcommand{\dx}{\mathrm{d}x}

% Alias for the Solution section header
\newcommand{\solution}{\textbf{\large Solution}}

% Probability commands: Expectation, Variance, Covariance, Bias
\newcommand{\E}{\mathrm{E}}
\newcommand{\Var}{\mathrm{Var}}
\newcommand{\Cov}{\mathrm{Cov}}
\newcommand{\Bias}{\mathrm{Bias}}

\begin{document}

\maketitle

\pagebreak

\begin{homeworkProblem}[6]
    Which of the following subsets of $\mathbb{R}$[$x]$ are subrings of $\mathbb{R}$[$x]$. 

    \begin{enumerate}[label=\alph*]
        \item All polynomials with constant term of $0_R$
        \item All polynomials of degree 2.
   
    \end{enumerate}

    \textbf{Solution}\\

    	 \textbf{a)} This is a subring. We will show that it is a nonempty subset closed under subtraction and multiplication. First, zero is a polynomial with the constant term $0_R$, so it is a nonempty subset. Next, consider polynomials $a,b \epsilon \mathbb{R}[x]$. Then if we subtract them and $0_R$, we get $(a+0_R) - (b+0_R) = (a-b) + (0_R-0_R) = a-b$, and since $a,b\epsilon{R}$, it is closed under subtraction. Similarly for multiplication, we get $(a+0_R)*(b+0_R) = (ab)+(0_R*b)+(0_R*a) + 0_R = ab$, so it is similarly closed under multiplication. Therefore, it is a subring of $\mathbb{R}[x]$.\\



		\textbf{b)} This is not a subring because it is not closed under multiplication. Consider the polynomials  $x^2$ and $x^2 +1$.When we multiply them, $x^2 * (x^2+1) = x^4 + x^2$, which is not a degree 2 polynomial, and therefore it is not closed under multiplication and cannot be a ring. Furthermore, $0_R$ is not of degree 2 and therefore is not in the subset. 


\end{homeworkProblem}

\begin{homeworkProblem}[20]
	Let $D:\mathbb{R}[x] \rightarrow \mathbb{R}[x]$ be the derivative map defined by 
	\[D(a_0 + a_1x +a_2x^2 + ... + a_nx^n) = a_1 + 2a_x + 3a_3x^2 + ... + na_nx^{n-1}\]	
	Is $D$ a homormorphism of rings? An isomorphism?\\
	
	\textbf{Solution}\\
	
	This is neither a homomorphism of rings nor an isomorphism. We will prove this by showing $D$ does not preserve multiplication. Consider $x^2, x^3 \epsilon \mathbb{R}[x]$. When we multiply them, $x^2 *x^3 = x^5$. Now, we will apply $D$ and multiply them. $D(x^2) * D(x^3) = 2x * 3x^2 = 6x^3 \neq x^5$. Therefore, $D$ does not preserve multiplication and cannot be a homomorphic or isomorphic.

\end{homeworkProblem}

\begin{homeworkProblem}[13]
	Prove Theorem 4.10.
	
	\textbf{Theorem 4.10} \\
	Let $\mathbb{F}$ be a field and $a(x), b(x), c(x) \epsilon \mathbb{F}[x]$. If $a(x)|b(x)c(x)$ and $a(x)$ and $b(x)$ are relatively prime, then $a(x)|c(x)$.\\
	
	\textbf{Solution}\\
	Assume $\mathbb{F}$ be a field and $a(x), b(x), c(x) \epsilon \mathbb{F}[x]$, $a(x)|b(x)c(x)$, and $a(x)$ and $b(x)$ are relatively prime. Then gcd($a(x), b(x)$)=1, by the definition of relatively prime. But then, by Theorem 4.8, this means that for polynomials $u(x), v(x) \epsilon{F}[x]$, 
	\[1 = a(x)v(x) + b(x)u(x).\]
	We can multiply this equation by $c(x)$ to get that
	\[ c(x)a(x)v(x) + c(x)b(x)u(x) = c(x)\]
Since $a(x)|b(x)c(x)$, there exists some $z(x) \epsilon \mathbb(F)[x]$ such that $a(x)z(x) = b(x)c(x)$. We can substitute this into the above equation to get 
	\[c(x)a(x)v(x) + a(x)z(x)u(x) = c(x)\]
and we can then factor out $a(x)$ to get 
	\[a(x)[c(x)v(x) + z(x)u(x)] = c(x)\]
Since $c(x), v(x), z(x), u(x) \epsilon \mathbb{F}[x]$, $a(x)|c(x)$.
\end{homeworkProblem}

\begin{homeworkProblem}[12]
	Express $x^4-4$ as a product of irreducibles in $\mathbb{Q}[x], \mathbb{R}[x],$ and in $\mathbb{C}[x]$.\\
	
	\textbf{Solution}\\
	First, we will factor this into a product of irreducibles in $\mathbb{C}[x]$. In $\mathbb{C}[x]$, 
	\[x^4-4 = (x-i\sqrt{2})(x+i\sqrt{2})(x-2)(x+2)\]
	These factors are all irreducible because they are all degree 1, both complex numbers and irrational numbers are in $\mathbb{C}$\\
In $\mathbb{R}[x]$,
	\[x^4-4 = (x^2+2)(x+\sqrt{2})(x-\sqrt{2})\]
In this case, $(x+\sqrt{2})$ and $(x-\sqrt{2})$ are factors because they are degree 1 with coefficients in the real numbers. $(x+\sqrt{2})$ is only irreducible in $\mathbb{C}[x]$ because it has complex coefficients.\\
In $\mathbb{Q}[x]$, 
	\[x^4-4 = (x^2+2)(x^2-2)\]
The first factor is irreducible because if it were not irreducible, it would be reducible in $\mathbb{R}[x]$. The second factor is irreducible because it only factors into polynomials with irrational coefficients, which, by definition, are not in the $\mathbb{Q}$
	
\end{homeworkProblem}

\begin{homeworkProblem}[4]
    \begin{enumerate}[label=\alph*]
        \item For what value of $k$ is $x-2$ a factor of $x^4-5x^3+5x^2+3x+k$ in $\mathbb{Q}[x]$
        \item For what value of $k$ is $x+1$ a factor of $x^4+2x^3-3x^2+kx+1$ in $\mathbb{Z}_5[x]$
    \end{enumerate}
        
        \textbf{Solution}\\
        \textbf{a)} For $k=-2$ is $(x-2)$ a factor of $x^4-5x^3+5x^2+3x+k$. This is simply to solve for because $(x-2)$ is in the form of a root, so we can evaluate our polynomial at $x=2$ and solve for $k$.
        \[2^4 -5x^3 + 5x^2+3x-k=0 \Rightarrow 16-40+20+6=-k \Rightarrow -2=k\]
        \textbf{b)} For $k=3$ is $x+1$ a factor of $x^4+2x^3-3x^2+kx+1$ in $\mathbb{Z}_5[x]$. Like in part a, since $(x+1)$ is in the form of a root, we can evaluate our polynomial at $x=-1$ and solve for $k$.
        \[(-1)^4+2(-1)^3-3(-1)^2 +k(-1) +1 =0 \Rightarrow 1-2-3-k+1=0 \Rightarrow 2=k+5 \Rightarrow k=3\]
   
    

\end{homeworkProblem}
\begin{homeworkProblem}[5]
\end{homeworkProblem}
\begin{homeworkProblem}[14]a
\end{homeworkProblem}
\begin{homeworkProblem}[18]
\end{homeworkProblem}
\begin{homeworkProblem}[19]
\end{homeworkProblem}

\pagebreak

\end{document}

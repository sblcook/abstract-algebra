\documentclass{article}

\usepackage{fancyhdr}
\usepackage{extramarks}
\usepackage{amsmath}
\usepackage{amsthm}
\usepackage{amsfonts}
\usepackage{tikz}
\usepackage[plain]{algorithm}
\usepackage{algpseudocode}
\usepackage{enumitem}
\usepackage{polynom}

\usetikzlibrary{automata,positioning}

%
% Basic Document Settings
%

\topmargin=-0.45in
\evensidemargin=0in
\oddsidemargin=0in
\textwidth=6.5in
\textheight=9.0in
\headsep=0.25in

\linespread{1.1}

\pagestyle{fancy}
\lhead{\hmwkAuthorName}
\chead{\hmwkClass}
\rhead{\firstxmark}
\lfoot{\lastxmark}
\cfoot{\thepage}

\renewcommand\headrulewidth{0.4pt}
\renewcommand\footrulewidth{0.4pt}

\setlength\parindent{0pt}

%
% Create Problem Sections
%

\newcommand{\enterProblemHeader}[1]{
    \nobreak\extramarks{}{Problem \arabic{#1} continued on next page\ldots}\nobreak{}
    \nobreak\extramarks{Problem \arabic{#1} (continued)}{Problem \arabic{#1} continued on next page\ldots}\nobreak{}
}

\newcommand{\exitProblemHeader}[1]{
    \nobreak\extramarks{Problem \arabic{#1} (continued)}{Problem \arabic{#1} continued on next page\ldots}\nobreak{}
    \stepcounter{#1}
    \nobreak\extramarks{Problem \arabic{#1}}{}\nobreak{}
}

\setcounter{secnumdepth}{0}
\newcounter{partCounter}
\newcounter{homeworkProblemCounter}
\setcounter{homeworkProblemCounter}{1}
\nobreak\extramarks{Problem \arabic{homeworkProblemCounter}}{}\nobreak{}

%
% Homework Problem Environment
%
% This environment takes an optional argument. When given, it will adjust the
% problem counter. This is useful for when the problems given for your
% assignment aren't sequential. See the last 3 problems of this template for an
% example.
%
\newenvironment{homeworkProblem}[1][-1]{
    \ifnum#1>0
        \setcounter{homeworkProblemCounter}{#1}
    \fi
    \section{Problem \arabic{homeworkProblemCounter}}
    \setcounter{partCounter}{1}
    \enterProblemHeader{homeworkProblemCounter}
}{
    \exitProblemHeader{homeworkProblemCounter}
}

%
% Homework Details
%   - Title
%   - Due date
%   - Class
%   - Section/Time
%   - Instructor
%   - Author
%

\newcommand{\hmwkTitle}{Homework\ \#5}
\newcommand{\hmwkDueDate}{November 13}
\newcommand{\hmwkClass}{Intro to Modern Algebra}
\newcommand{\hmwkClassTime}{9:30-11:00}
\newcommand{\hmwkClassInstructor}{Professor Lorenz}
\newcommand{\hmwkAuthorName}{\textbf{Sam Cook}}

%
% Title Page
%

\title{
    \vspace{2in}
    \textmd{\textbf{\hmwkClass:\ \hmwkTitle}}\\
    \normalsize\vspace{0.1in}\small{Due\ on\ \hmwkDueDate\ at 8:00 am}\\
    \vspace{0.1in}\large{\textit{\hmwkClassInstructor\ \hmwkClassTime}}
    \vspace{3in}
}

\author{\hmwkAuthorName}
\date{}

\renewcommand{\part}[1]{\textbf{\large Part \Alph{partCounter}}\stepcounter{partCounter}\\}

%
% Various Helper Commands
%


% Useful for algorithms
\newcommand{\alg}[1]{\textsc{\bfseries \footnotesize #1}}

% For derivatives
\newcommand{\deriv}[1]{\frac{\mathrm{d}}{\mathrm{d}x} (#1)}

% For partial derivatives
\newcommand{\pderiv}[2]{\frac{\partial}{\partial #1} (#2)}

% Integral dx
\newcommand{\dx}{\mathrm{d}x}

% Alias for the Solution section header
\newcommand{\solution}{\textbf{\large Solution}}

% Probability commands: Expectation, Variance, Covariance, Bias
\newcommand{\E}{\mathrm{E}}
\newcommand{\Var}{\mathrm{Var}}
\newcommand{\Cov}{\mathrm{Cov}}
\newcommand{\Bias}{\mathrm{Bias}}

\begin{document}

\maketitle

\pagebreak

\begin{homeworkProblem}
	Factor each polynomial into a product of irreducible polynomials. Be sure to justify why each factor is irreducible.
	
	\begin{enumerate}
        \item $f(x) = 2x^4-5x^3+3x^2+4x-6$ in $\mathbb{Q}[x]$
        \item $g(x) = x^3 = x^2 + 5x + 5$ in $\mathbb{Z}_7[x]$
    \end{enumerate}
    
    \textbf{Solution}\\
    
    	\textbf{1)} The polynomial written as a product of irreducibles is
    	\[(x+1)(x-3/2)(2x^2-4x-4)\]
    	$(x+1)$ and $(x-3/2)$ are irreducible because they both are degree 1, and by definition are irreducible. Since $2x^2-4x-4$ is degree 2, it has roots if and only if it is reducible. These two factors were found by using the Rational Root Theorem. Futhermore, by the Rational Root Theorem, we know that the only possible roots in $\mathbb{Q}$ are $\pm \{1, 2\}$. However, none of these are roots as $f(1) = -6, f(-1) = 2, f(2) = -4, f(-2) = 12$, none of which are 0, and thus, it has no roots and is irreducible. Therefore, this is a product of irreducibles.\\
    	
    	\textbf{2)} The polynomial written as a product of irreducibles is
    	\[g(x) = (x+4)(x+1)(x+3)\]
    	We know that all three factors are irreducible because they are all degree 1. This solution was obtained by simply testing to see if any of the 7 elements of $\mathbb{Z}_7$ are roots, since this is a degree 3 polynomial that factors if and only if it has a root. In our case, $g(4) = 0, g(1) = 0, g(3) = 0$.

\end{homeworkProblem}


\pagebreak

\end{document}

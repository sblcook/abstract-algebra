\documentclass[a4paper,11pt]{article}
\usepackage{amsmath}
\usepackage{amssymb}
\usepackage[margin = 1.0in]{geometry}
\usepackage{polynom}
% define the title
\author{Sam Cook}
\title{Intro to Modern Algebra \\Recitation 7}
\date{October 17th, 2017}
\begin{document}
% generates the title
\maketitle

\paragraph{Step 3:\\}
Let $f(x) = 3x^3 + x^2 + 2x$ and $g(x) = 2x^2+x+4$
\subparagraph{b)}
If $f$ and $g$ are elements of $\mathbb{Z}_5[x]$, use long division to find $q(x)$ and $r(x)$ so that $f(x) = g(x)q(x) + r(x)$.\\

%\polylongdiv{3x^3 + x^2 + 2x + 2}{2x^2+x+4}

%$$
%\begin{array}{rc@{}c}
%& 1 & 3 \\ \cline{2-3}
%\multicolumn{1}{r|}{7} & 9 & 1 \\
%& 7 & 0 \\ \cline{2-3}
%& 2 & 1 \\ 
%& 2 & 1 \\ \cline{2-3}
%& & 0
%\end{array}
%$$
Long division:

$$
\begin{array}{rc@{}c@{} c@{}c}
&&& 4x & +1\\ \cline{2-5}
\multicolumn{1}{r|}{2x^2 + x + 4} & 3x^3& +x^2&+2x& +2\\
&-3x^3&-4x^2 &-x&\\ \cline{2-5}
&0&+2x^2&+x& +2\\
&&-2x^2&-x&-4\\ \cline{3-5}
&& 0 & +0& +3\\
\end{array}
$$

Therefore, $q(x) = 4x+1$ and $r(x) = 3$ in the form $f(x) = g(x)q(x) + r(x)$.


\subparagraph{c)}
What happens if you try to do the same thing viewing $f$ and $f$ as elements of $\mathbb{Z}_6$?\\

When trying to do the long division, one immediately runs into a problem when trying to find a solution to $2x = 3$. This occurs in the first step trying to cancel out the dominant term in the polynomial $f(x) = 3x^3 + x^2 + 2x$. This is (probably) because $\mathbb{Z}_6$ is a ring, while $\mathbb{Z}_5$ is a field.



\end{document}